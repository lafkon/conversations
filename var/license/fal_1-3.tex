\documentclass[8pt,a4paper]{article}

\title{Free Art License 1.3}

\usepackage{multicol}
\usepackage{scalefnt}
\usepackage{paralist}

\setlength\textheight{280mm}
\setlength\textwidth{190mm}
\setlength\topmargin{-25mm}
\setlength\oddsidemargin{-15mm}
\setlength\itemsep{-20mm}
\setdefaultleftmargin{0.5em}{0.5em}{}{}{}{}
\pagestyle{empty}
\parindent=0pt
\renewcommand{\rmdefault}{pju}

\newcommand\h[1]{\medskip{\fontfamily{ocr}\selectfont{\scalefont{.9}#1}}}

\begin{document}
%
\scalefont{.95}%
\begin{multicols}{3}[][.4\paperwidth]%
{\scalefont{1.2}%
\textbf{%
Free Art License 1.3.
(C) Copyleft Attitude, 2007.
You can make reproductions and distribute this license verbatim
(without any changes).
Translation: Jonathan Clarke, Benjamin Jean, Griselda Jung, 
Fanny Mourguet, Antoine Pitrou.
Thanks to framalang.org
}}

\bigskip

\h{PREAMBLE}

The Free Art License grants the right to freely copy, distribute, and transform 
creative works without infringing the author's rights.

The Free Art License recognizes and protects these rights. Their implementation 
has been reformulated in order to allow everyone to use creations of the human 
mind in a creative manner, regardless of their types and ways of expression.

While the public's access to creations of the human mind usually is restricted 
by the implementation of copyright law, it is favoured by the Free Art License. 
This license intends to allow the use of a work’s resources; to establish new 
conditions for creating in order to increase creation opportunities. The Free 
Art License grants the right to use a work, and acknowledges the right 
holder’s and the user’s rights and responsibility.

The invention and development of digital technologies, Internet and Free 
Software have changed creation methods: creations of the human mind can 
obviously be distributed, exchanged, and transformed. They allow to produce 
common works to which everyone can contribute to the benefit of all.

The main rationale for this Free Art License is to promote and protect these 
creations of the human mind according to the principles of copyleft: freedom to 
use, copy, distribute, transform, and prohibition of exclusive appropriation. 


\h{DEFINITIONS}

``\emph{work}'' either means the initial work, the subsequent works or the 
common work as defined hereafter:

``\emph{common work}'' means a work composed of the initial work and all 
subsequent contributions to it (originals and copies). The initial author is 
the one who, by choosing this license, defines the conditions under which 
contributions are made.

``\emph{Initial work}'' means the work created by the initiator of the common 
work (as defined above), the copies of which can be modified by whoever wants to

``\emph{Subsequent works}'' means the contributions made by authors who 
participate in the evolution of the common work by exercising the rights to 
reproduce, distribute, and modify that are granted by the license.

``\emph{Originals}'' (sources or resources of the work) means all copies of 
either the initial work or any subsequent work mentioning a date and used by 
their author(s) as references for any subsequent updates, interpretations, 
copies or reproductions.

``\emph{Copy}'' means any reproduction of an original as defined by this 
license. 


\h{OBJECT}

The aim of this license is to define the conditions under which one can use 
this work freely. 


\h{SCOPE}

This work is subject to copyright law. Through this license its author 
specifies the extent to which you can copy, distribute, and modify it.


\h{FREEDOM TO COPY (OR TO MAKE REPRODUCTIONS)}

You have the right to copy this work for yourself, your friends or any other 
person, whatever the technique used. 


\h{FREEDOM TO DISTRIBUTE, TO PERFORM IN PUBLIC}

You have the right to distribute copies of this work; whether modified or not, 
whatever the medium and the place, with or without any charge, provided that 
you:
attach this license without any modification to the copies of this work or 
indicate precisely where the license can be found,
specify to the recipient the names of the author(s) of the originals, including 
yours if you have modified the work,
specify to the recipient where to access the originals (either initial or 
subsequent).
The authors of the originals may, if they wish to, give you the right to 
distribute the originals under the same conditions as the copies.


\h{FREEDOM TO MODIFY}

You have the right to modify copies of the originals (whether initial or 
subsequent) provided you comply with the following conditions:
all conditions in article 2.2 above, if you distribute modified copies;
indicate that the work has been modified and, if it is possible, what kind of 
modifications have been made;
distribute the subsequent work under the same license or any compatible license.
The author(s) of the original work may give you the right to modify it under 
the same conditions as the copies. 


\h{RELATED RIGHTS}

Activities giving rise to author’s rights and related rights shall not 
challenge the rights granted by this license.
For example, this is the reason why performances must be subject to the same 
license or a compatible license. Similarly, integrating the work in a database, 
a compilation or an anthology shall not prevent anyone from using the work 
under the same conditions as those defined in this license.


\h{INCORPORATION OF THE WORK}

Incorporating this work into a larger work that is not subject to the Free Art 
License shall not challenge the rights granted by this license.
If the work can no longer be accessed apart from the larger work in which it is 
incorporated, then incorporation shall only be allowed under the condition that 
the larger work is subject either to the Free Art License or a compatible 
license.


\h{COMPATIBILITY}

A license is compatible with the Free Art License provided:
it gives the right to copy, distribute, and modify copies of the work including 
for commercial purposes and without any other restrictions than those required 
by the respect of the other compatibility criteria;
it ensures proper attribution of the work to its authors and access to previous 
versions of the work when possible;
it recognizes the Free Art License as compatible (reciprocity);
it requires that changes made to the work be subject to the same license or to 
a license which also meets these compatibility criteria.


\h{YOUR INTELLECTUAL RIGHTS}

This license does not aim at denying your author's rights in your contribution 
or any related right. By choosing to contribute to the development of this 
common work, you only agree to grant others the same rights with regard to your 
contribution as those you were granted by this license. Conferring these rights 
does not mean you have to give up your intellectual rights.


\h{YOUR RESPONSIBILITIES}

The freedom to use the work as defined by the Free Art License (right to copy, 
distribute, modify) implies that everyone is responsible for their own actions.


\h{DURATION OF THE LICENSE}

This license takes effect as of your acceptance of its terms. The act of 
copying, distributing, or modifying the work constitutes a tacit agreement. 
This license will remain in effect for as long as the copyright which is 
attached to the work. If you do not respect the terms of this license, you 
automatically lose the rights that it confers.
If the legal status or legislation to which you are subject makes it impossible 
for you to respect the terms of this license, you may not make use of the 
rights which it confers.


\h{VARIOUS VERSIONS OF THE LICENSE}

This license may undergo periodic modifications to incorporate improvements by 
its authors (instigators of the “Copyleft Attitude” movement) by way of 
new, numbered versions.
You will always have the choice of accepting the terms contained in the version 
under which the copy of the work was distributed to you, or alternatively, to 
use the provisions of one of the subsequent versions. 


\h{SUB-LICENSING}

Sub-licenses are not authorized by this license. Any person wishing to make use 
of the rights that it confers will be directly bound to the authors of the 
common work.


\h{LEGAL FRAMEWORK}

This license is written with respect to both French law and the Berne 
Convention for the Protection of Literary and Artistic Works.

\end{multicols}

\end{document}

